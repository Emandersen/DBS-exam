\documentclass[11pt,a4paper]{article}

% Packages
\usepackage[utf8]{inputenc}
\usepackage[english]{babel}
\usepackage{amsmath,amssymb}
\usepackage{geometry}
\usepackage{hyperref}
\usepackage{enumitem}
\usepackage{graphicx}
\usepackage{booktabs}

% Page setup
\geometry{margin=2.5cm}
\hypersetup{
    colorlinks=true,
    linkcolor=blue,
    urlcolor=blue,
    citecolor=blue
}

% Title information
\title{DBS Exam Notes}
\author{}
\date{\today}

\begin{document}

\maketitle

\tableofcontents
\newpage

\section{Functional Dependencies}

\subsection{The "Boss and Follower" Logic}
A functional dependency $\alpha \to \beta$ is a rule stating that if two rows have the same value for the "Boss" ($\alpha$), they \textbf{must} have the same value for the "Follower" ($\beta$).

\subsubsection*{Case Study: Slide 16 Instance}
Based on the table instance provided on Slide 16 of Lecture 7, here is the verification for each functional dependency using the step-by-step process:

\begin{center}
\begin{tabular}{@{}cccc@{}}
\toprule
\textbf{A} & \textbf{B} & \textbf{C} & \textbf{D} \\ \midrule
a1 & b1 & c1 & d1 \\
a2 & b2 & c2 & d2 \\
a2 & b2 & c2 & d3 \\
a3 & b1 & c3 & d3 \\ \bottomrule
\end{tabular}
\end{center}

\subsection{Verification Results}

\begin{itemize}
    \item \textbf{$A \to B$ (Holds):}
    \begin{enumerate}
        \item \textbf{Step-by-Step Check:} Identify the Boss ($A$) and Follower ($B$).
        \item \textbf{Find Duplicates:} Look for rows where the Boss ($A$) has the same value.
        \item \textbf{Check Followers:} For each duplicate Boss, verify that the Follower ($B$) values are identical.
        \item \textbf{Verification:} The only duplicate "Boss" in column $A$ is \textbf{a2} (rows 2 and 3).
        \item \textbf{Verdict:} In both rows, the "Follower" in column $B$ is \textbf{b2}. Since the followers are identical for the duplicate boss, the dependency holds.
    \end{enumerate}

    \item \textbf{$A \to C$ (Holds):}
    \begin{itemize}
        \item For the duplicate boss \textbf{a2}, both rows have the identical follower \textbf{c2} in column $C$.
    \end{itemize}

    \item \textbf{$A \to D$ (Fails):}
    \begin{itemize}
        \item For the duplicate boss \textbf{a2}, the followers in column $D$ are \textbf{d2} and \textbf{d3}.
        \item Because the followers are different for the same boss, the dependency is broken.
    \end{itemize}
\end{itemize}

\section{Finding Superkeys}

\subsection{Step-by-Step Process}
\begin{enumerate}
    \item Start with a candidate attribute (e.g., $B$)
    \item Check if $B$ is a "Boss" for any functional dependency rules
    \item If $B \to A$, your set becomes $\{B, A\}$
    \item Continue using your new set to unlock more attributes
    \item If $\{A, B\}$ is now in your set and $AB \to C$, you add $C$
    \item If you reach all attributes $\{A, B, C, D, E, F\}$, it is a superkey
\end{enumerate}

\section{Normal Form Audit}

\subsection{BCNF Check}
\begin{itemize}
    \item Look at every functional dependency
    \item Is the "Boss" (left side) a superkey?
    \item If even one is not, it is \textbf{NOT in BCNF}
\end{itemize}

\subsection{3NF Check}
\begin{itemize}
    \item If BCNF fails, check the "Follower" (right side)
    \item Is it a prime attribute (part of any candidate key)?
    \item If yes, it is \textbf{3NF}
\end{itemize}

\section{Lossless Join}

A split into $R_1$ and $R_2$ is \textbf{lossless} if the attributes they share are a superkey for at least one of the two resulting tables.

\section{External Merge Sort: 2-Way Algorithm}

\subsection{Problem Example}
The answer to Question 6.1 is $\lceil \log_2 2{,}000 \rceil$ because the algorithm must first convert the raw data into manageable pages and then iteratively merge those pages until they are sorted.

\subsection{Step-by-Step Breakdown}

\subsubsection*{Step 1: Calculate the Number of Pages ($B$)}
The algorithm operates on pages (blocks), not individual records.
\begin{itemize}
    \item Total Tuples ($n_{r1}$): 100,000
    \item Tuples per Page: 50
    \item Total Pages ($B$): $\frac{100{,}000}{50} = 2{,}000$ pages
\end{itemize}

\subsubsection*{Step 2: Understand the Sorting Phases}
External sorting is divided into two distinct phases:
\begin{itemize}
    \item \textbf{Phase 1 (Pass 0):} The database reads each page into memory, sorts it, and writes it back to disk. This creates 2,000 sorted runs, each consisting of 1 page.
    \item \textbf{Phase 2 (The Merge Phase):} This is what the question specifically asks for. In this phase, the algorithm takes the sorted runs and merges them into larger and larger runs.
\end{itemize}

\subsubsection*{Step 3: Apply the 2-Way Merge Logic}
In a 2-Way merge, the computer uses 3 buffer pages: two for input (to read two runs) and one for output (to write the merged result).

\textbf{The Power of 2:} Because it is a "2-Way" merge, it combines 2 runs into 1 larger run during every pass.
\begin{itemize}
    \item Pass 1: 2,000 runs are merged into 1,000 runs
    \item Pass 2: 1,000 runs are merged into 500 runs
    \item \textbf{Goal:} This continues until only 1 single sorted run remains
\end{itemize}

\subsubsection*{Step 4: The Mathematical Formula}
To find out how many times you must halve the number of runs to reach 1, you use a logarithm with base 2.

\textbf{Formula for Phase 2 Passes:} $\lceil \log_2 (\text{Initial Runs}) \rceil$

Since Phase 1 produced 2,000 runs, Phase 2 requires $\lceil \log_2 2{,}000 \rceil$ passes.

\subsection{Why the Other Options Are Wrong}
\begin{itemize}
    \item \textbf{(a) $\lceil \log_2 100{,}000 \rceil$:} This uses the number of tuples, but the database sorts pages.
    \item \textbf{(c) \& (d):} These use a base of 299 ($M-1$), which is the formula for a Multi-Way Merge Sort using all 300 buffer pages, but the question explicitly asked for the 2-Way algorithm.
\end{itemize}

\end{document}
